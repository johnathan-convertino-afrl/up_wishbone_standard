\begin{titlepage}
  \begin{center}

  {\Huge uP Wishbone Standard}

  \vspace{25mm}

  \includegraphics[width=0.90\textwidth,height=\textheight,keepaspectratio]{img/AFRL.png}

  \vspace{25mm}

  \today

  \vspace{15mm}

  {\Large Jay Convertino}

  \end{center}
\end{titlepage}

\tableofcontents

\newpage

\section{Usage}

\subsection{Introduction}

\par
This core converts the Wishbone Standard bus to the uP bus. This allows any core with a uP bus to be interfaced with a Wishbone Standard bus. Combination of combinatorial logic and synchronous logic.

\subsection{Dependencies}

\par
The following are the dependencies of the cores.

\begin{itemize}
  \item fusesoc 2.X
  \item iverilog (simulation)
  \item cocotb (simulation)
\end{itemize}

\subsubsection{fusesoc\_info Depenecies}
\begin{itemize}
\item dep
	\begin{itemize}
	\item AFRL:utility:helper:1.0.0
	\end{itemize}
\end{itemize}


\subsection{In a Project}
\par
This core is made to interface Wishbone Standard bus to uP based device cores. This is part of a family of converters based on Analog Devices uP specification. Using this allows usage of Analog Devices AXI Lite core, AFRL APB3, AFRL Wishbone Standard, and AFRL Wishbone Pipeline converters. Meaning any uP core can be easily customized to any bus quickly. These are made for relativly slow speed bus device interfaces. An example of a Verilog uP interface provided below.

\begin{lstlisting}[language=Verilog]
  //output signals assigned to registers.
  assign up_rack  = r_up_rack & up_rreq;
  assign up_wack  = r_up_wack & up_wreq;
  assign up_rdata = r_up_rdata;
  assign irq      = r_irq;

  assign s_rx_ren = ((up_raddr[3:0] == RX_FIFO_REG) && up_rreq ? r_up_rack & r_rx_ren : 0);

  //up registers decoder
  always @(posedge clk)
  begin
    if(rstn == 1'b0)
    begin
      r_up_rack   <= 1'b0;
      r_up_wack   <= 1'b0;
      r_up_rdata  <= 0;

      r_rx_ren    <= 1'b0;

      r_overflow  <= 1'b0;

      r_control_reg <= 0;
    end else begin
      r_up_rack   <= 1'b0;
      r_up_wack   <= 1'b0;
      r_tx_wen    <= 1'b0;
      r_rx_ren    <= 1'b0;
      r_up_rdata  <= r_up_rdata;
      //clear reset bits
      r_control_reg[RESET_RX_BIT] <= 1'b0;
      r_control_reg[RESET_TX_BIT] <= 1'b0;

      if(rx_full == 1'b1)
      begin
        r_overflow <= 1'b1;
      end

      //read request
      if(up_rreq == 1'b1)
      begin
        r_up_rack <= 1'b1;

        case(up_raddr[3:0])
          RX_FIFO_REG: begin
            r_up_rdata <= rx_rdata & {{(BUS_WIDTH*8-DATA_BITS){1'b0}}, {DATA_BITS{1'b1}}};
            r_rx_ren <= 1'b1;
          end
          STATUS_REG: begin
            r_up_rdata <= {{(BUS_WIDTH*8-8){1'b0}}, s_parity_err, s_frame_err, r_overflow, r_irq_en, tx_full, tx_empty, rx_full, rx_valid};
            r_overflow <= 1'b0;
          end
          default:begin
            r_up_rdata <= 0;
          end
        endcase
      end

      //write request
      if(up_wreq == 1'b1)
      begin
        r_up_wack <= 1'b1;

        //only allow write once ack (Analog Devices does the same)
        if(r_up_wack == 1'b1) begin
          case(up_waddr[3:0])
            TX_FIFO_REG: begin
              r_tx_wdata  <= up_wdata;
              r_tx_wen    <= 1'b1;
            end
            CONTROL_REG: begin
              r_control_reg <= up_wdata;
            end
            default:begin
            end
          endcase
        end
      end
    end
  end

  //up control register processing and fifo reset
  always @(posedge clk)
  begin
    if(rstn == 1'b0)Classic
    begin
      r_rstn_rx_delay <= ~0;
      r_rstn_tx_delay <= ~0;
      r_irq_en <= 1'b0;
    end else begin
      r_rstn_rx_delay <= {1'b1, r_rstn_rx_delay[FIFO_DEPTH-1:1]};
      r_rstn_tx_delay <= {1'b1, r_rstn_rx_delay[FIFO_DEPTH-1:1]};

      if(r_control_reg[RESET_RX_BIT])
      begin
        r_rstn_rx_delay <= {FIFO_DEPTH{1'b0}};
      end

      if(r_control_reg[RESET_TX_BIT])
      begin
        r_rstn_tx_delay <= {FIFO_DEPTH{1'b0}};
      end

      if(r_control_reg[ENABLE_INTR_BIT] != r_irq_en)
        r_irq_en <= r_control_reg[ENABLE_INTR_BIT];
      end
    end
  end
\end{lstlisting}

\section{Architecture}
\par
The only module is the up\_wishbone\_standard module. It is listed below.

\begin{itemize}
  \item \textbf{up\_wishbone\_standard} Convert Wishbone Standard to the Analog Devices uP BUS. (see core for documentation).
\end{itemize}

\par
This core has two generate blocks, and two always blocks. They are listed below.

generate:
\begin{enumerate}
\item Part Select Write generates uP signals that have the non-selected elements blanked out with zeros.
\item Part Select Read generates Wishbone Standard signals that have non-selected elements blanked out with zeros.
\end{enumerate}

Please see \ref{Module Documentation} for more information.

\section{Building}

\par
The Wishbone Standard core is written in Verilog 2001. They should synthesize in any modern FPGA software. The core comes as a fusesoc packaged core and can be
included in any other core. Be sure to make sure you have meet the dependencies listed in the previous section.

\subsection{fusesoc}
\par
Fusesoc is a system for building FPGA software without relying on the internal project management of the tool. Avoiding vendor lock in to Vivado or Quartus.
These cores, when included in a project, can be easily integrated and targets created based upon the end developer needs. The core by itself is not a part of
a system and should be integrated into a fusesoc based system. Simulations are setup to use fusesoc and are a part of its targets.

\subsection{Source Files}

\subsubsection{fusesoc\_info File List}
\begin{itemize}
\item src
	\begin{itemize}
	\item src/up\_wishbone\_standard.v
	\end{itemize}
\item tb
	\begin{itemize}
	\item tb/tb\_wishbone\_slave.v
	\end{itemize}
\item tb\_cocotb
	\begin{itemize}
	\item {'tb/tb\_cocotb.py': {'file\_type': 'user', 'copyto': '.'}}
	\item {'tb/tb\_cocotb.v': {'file\_type': 'verilogSource'}}
	\end{itemize}
\end{itemize}


\subsection{Targets}

\subsubsection{fusesoc\_info Targets}
\begin{itemize}
\item default
	\begin{itemize}
	\item[$\space$] Info: Default for IP intergration.
	\end{itemize}
\item sim
	\begin{itemize}
	\item[$\space$] Info: Base simulation using icarus as default.
	\end{itemize}
\end{itemize}


\subsection{Directory Guide}

\par
Below highlights important folders from the root of the directory.

\begin{enumerate}
  \item \textbf{docs} Contains all documentation related to this project.
    \begin{itemize}
      \item \textbf{manual} Contains user manual and github page that are generated from the latex sources.
      \item \textbf{specs} Contains specifications for the bus.
    \end{itemize}
  \item \textbf{src} Contains source files for the core
  \item \textbf{tb} Contains test bench files for iverilog and cocotb
    \begin{itemize}
      \item \textbf{cocotb} testbench files
    \end{itemize}
\end{enumerate}

\newpage

\section{Simulation}
\par
There are a few different simulations that can be run for this core.

\subsection{iverilog}
\par
iverilog is used for simple test benches for quick verification, visually, of the core.

\subsection{cocotb}
\par
To use the cocotb tests you must install the following python libraries.
\begin{lstlisting}[language=bash]
  $ pip install cocotb
  $ pip install cocotbext-up
  $ pip install cocotbext-wishbone
\end{lstlisting}

Each module has a cocotb based simulation. These use the cocotb extensions Wishbone and uP. To install
these locally use the following.cocotb
\begin{lstlisting}[language=bash]
  $ pip install --break-system-packages -e .
\end{lstlisting}

\begin{itemize}
  \item \textbf{sim\_cocotb} Standard simulation Wishbone to uP conversion using cocotbexts.
\end{itemize}

Then you must use the cocotb sim target. The targets above can be run with various parameters.
\begin{lstlisting}[language=bash]
  $ fusesoc run --target sim_cocotb AFRL:bus:up_wishbone_standard:1.0.0
\end{lstlisting}

\newpage

\section{Module Documentation} \label{Module Documentation}

\par

\begin{itemize}
\item \textbf{up\_wishbone\_standard} Wishbone to uP converter\\
\item \textbf{tb\_wishbone\_standard-v} Verilog test bench\\
\item \textbf{tb\_cocotb-py} Cocotb python test routines\\
\item \textbf{tb\_cocotb-v} Cocotb verilog test bench\\
\end{itemize}
The next sections document the module in detail.

